\documentclass[a4paper,12pt]{article}
\usepackage[utf8]{inputenc}
\usepackage{amsmath}
\usepackage{amsfonts}
\usepackage{amssymb}
\usepackage{graphicx}
\usepackage{listings}
\usepackage{color}
\usepackage{hyperref}
\usepackage[romanian]{babel}

\title{Stomadmin - Managementul unui cabinet stomatologic}
\author{Alexandru Munteanu\\
Darius Petrov\\
Ioan Samuilă\\
Departamentul de Informatică\\
Facultatea de Matematică și Informatică\\
Universitatea de Vest din Timișoara}

\begin{document}

\maketitle
\begin{abstract}
Stomadmin este o aplicație de managemet a fluxurilor de activități dintr-un cabinet stomatologic.

 În principal, gestionează evidența pacienților și a consultațiilor acordate acestora din punct de vedere al diagnosticării și medicației prescrise. 

Păstrează o evidența a consultațiilor, trimiterilor și a altor documente medicale cum ar fi de exemplu, scrisorile medicale.  
\end{abstract}

\pagebreak

\tableofcontents

\pagebreak

\section{Introducere}

Stomadmin este o aplicație dezvoltată cu ajutorul tehnologiilor web și utilizează pentru rulare sistemul client - server.

Într-o bază de date sunt păstrate înregistrările ce privesc activitățile medicale specifice unui cabinet stomatologic, iar această bază de date este găzduită de un server care se poate afla în locația clientului sau într-o alta locație.

Principalele tehnologiile folosite în procesul de dezvoltare sunt:

\begin{itemize}
\item PHP 7
\item Bootstrap 4
\item Laravel 5.x
\item MySQL
\end{itemize}

Accesarea datelor se face cu ajutorul unui browser web și permite diferite niveluri de acces. În principal vorbim de următoarele tipuri de conturi:

\begin{itemize}
\item Administrator
\item Medic Specialist
\item Secretară
\item Pacient
\end{itemize} 

\section{Functionalității}

Principalele funcționalității sunt:
\begin{itemize}
\item Grupuri de utilizatori după cum urmează:
\begin{itemize}
\item Administrator
\item Medic Specialist
\item Secretară
\item Pacient
\end{itemize}
\item Sistem de magement al fluxului de pacienți
\end{itemize}

\subsection{Administrator}

Administratorul are control deplin asupra sistemului.

Drepturile administratorilor:

\begin{itemize}
\item 1
\item 2
\item 3
\item Pot adăuga, șterge sau bloca utilizatori.
\end{itemize}

\subsection{Medic Specialist}

Următoarele specificații:

\begin{itemize}
\item 1
\item 2
\item 3
\item 4
\item 5
\item 6
\end{itemize}

\subsection{Secretara}

Următoarele specificații:

\begin{itemize}
\item 1
\item 2
\item 3
\item 4
\item 5
\item 6
\end{itemize}

\subsection{Pacient}

Următoarele specificații:

\begin{itemize}
\item 1
\item 2
\item 3
\item 4
\item 5
\item 6
\end{itemize}

\section{Concluzii}

Aspecte:

\begin{itemize}
\item 1
\item 2
\item 3
\item 4
\item 5
\item 6
\end{itemize}

\end{document}