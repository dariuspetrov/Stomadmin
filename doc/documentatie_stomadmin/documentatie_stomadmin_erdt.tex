\documentclass[a4paper,12pt]{article}
\usepackage[utf8]{inputenc}
\usepackage{amsmath}
\usepackage{amsfonts}
\usepackage{amssymb}
\usepackage{graphicx}
\usepackage{listings}
\usepackage{color}
\usepackage{hyperref}
\usepackage[romanian]{babel}

\title{Stomadmin - Managementul unui cabinet stomatologic}
\author{Alexandru Munteanu\\
Darius Petrov\\
Ioan Samuilă\\
Departamentul de Informatică\\
Facultatea de Matematică și Informatică\\
Universitatea de Vest din Timișoara}

\begin{document}

\maketitle
\begin{abstract}
Test rezumat
\end{abstract}

\pagebreak

\tableofcontents

\pagebreak

\section{Introducere}



\section{Functionalității}

Principalele funcționalității sunt:
\begin{itemize}
\item Grupuri de utilizatori după cum urmează:
\begin{itemize}
\item Administrator
\item Medic Specialist
\item Secretară
\item Pacient
\end{itemize}
\item Sistem de magement al fluxului de pacienți
\end{itemize}

\subsection{Administrator}

Administratorul are control deplin asupra sistemului.

Drepturile administratorilor:

\begin{itemize}
\item 1
\item 2
\item 3
\item Pot adăuga, șterge sau bloca utilizatori.
\end{itemize}

\subsection{Medic Specialist}

Următoarele specificații:

\begin{itemize}
\item 1
\item 2
\item 3
\item 4
\item 5
\item 6
\end{itemize}

\subsection{Secretara}

Următoarele specificații:

\begin{itemize}
\item 1
\item 2
\item 3
\item 4
\item 5
\item 6
\end{itemize}

\subsection{Pacient}

Următoarele specificații:

\begin{itemize}
\item 1
\item 2
\item 3
\item 4
\item 5
\item 6
\end{itemize}

\section{Concluzii}

Aspecte:

\begin{itemize}
\item 1
\item 2
\item 3
\item 4
\item 5
\item 6
\end{itemize}

\end{document}