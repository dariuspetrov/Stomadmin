\documentclass[a4paper,12pt]{article}
\usepackage[utf8]{inputenc}
\usepackage{amsmath}
\usepackage{amsfonts}
\usepackage{amssymb}
\usepackage{graphicx}
\usepackage{listings}
\usepackage{color}
\usepackage{hyperref}
\usepackage[romanian]{babel}

\title{Progresul Proiectului\\
Stomadmin - Managementul unui cabinet stomatologic}
\author{Alexandru Munteanu\\
Darius Petrov\\
Ioan Samuilă\\
Departamentul de Informatică\\
Facultatea de Matematică și Informatică\\
Universitatea de Vest din Timișoara}
\date{}

\begin{document}

\maketitle

\pagebreak

Proiectul a fost dezvoltat in principal prin parcurgera următorilor pași:

\begin{itemize}
\item Intervalul din săptămânile 0  - 7
\begin{itemize}
\item Alegerea proiectului;
\item studierea aplicațiilor similare, documentarea tipului de activitate ce urmează a fi modelat;
\item realizarea unei schițe cu funcționalitățile minimale;
\item simularea interacțiunii cu utilizatorul;
\item selectarea listei preliminare de tehnologii folosite;
\item organizarea echipei, distribuirea sarcinilor, inclusiv redistribuirea acestora în cazul unei probleme a echipei.
\end{itemize}
\item Intervalul din săptămânile 7 - 11
\begin{itemize}
\item Stabilirea tehnologiilor ce urmeaza a fi folosite;
\item dezvoltarea schiței initiale din care a rezultat etapa următoare;
\item dezvoltarea arhitecturii aplicației. Crearea diagramei de clase și a diagramei UML;
\item implementarea minimală a plicației având funcționalitătile de bază și testarea acestora.
\end{itemize}
\item Intervalul din săptămânile 11 - 14
\begin{itemize}
\item Dezvoltarea aplicației. Implemetarea incrementală a cerințelor utilizatorului.
\item Testarea finală a tuturor modulelor din punct de vedere al funcționalității și al interacțiunii intre acestea;
\item Elaborarea documentației.
\end{itemize}
\end{itemize}

\end{document}